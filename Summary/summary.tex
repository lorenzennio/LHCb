\documentclass[10pt]{article} 
\usepackage[top=2cm, bottom=2cm, left=2cm, right=2cm]{geometry} %This sets the margins of the report.

\usepackage{graphicx} % A package allowing insertion of images into the text.

\usepackage{color} % Allows the colour of the font to be changed by using the '\color' command: This is just to support the blue comments in this template...use standard (black) text in your report.

\linespread{-1} % Sets the spacing between lines of text.
\setlength{\parindent}{0cm}  % Suppresses indentation of text at the start of a paragraph

\begin{document} % This begins the document proper and ends the pre-amble
\title{LHCb summary}
\date{}
\maketitle
\section{Week 1}
\begin{itemize}
\item test data to analyse invariant mass of B mason assuming $KKK$ decay, using known kaon mass
\item looked at kaon/pion probabilities
\item started analysing $B^{\pm} \rightarrow K^{\pm}\pi^+\pi^-$ decay
\begin{itemize}
\item selection: only include $P(kaon) > 0.9$ and $P(pion) > 0.7$
\item selection: also excluded all muons
\item assigned new variables for pion and kaon variables
\item selection: pion charge sums to 0
\item found invariant mass of B mason with same technique as before
\end{itemize}
\item looked at two body resonance - intermediate decay via neutral particle
\begin{itemize}
\item want to get rid of  D mason decay as it decays via c quark and we are interested in b quark
\item selected data of B mason mass $\pm 60~MeV/c^2$ from mass plot 
\item three possible intermediate decays: $\pi^+\pi^-$, $K^+\pi^+$ and $K^-\pi^+$
\item computed invariant mass for all three possible cases
\item plot under condition that charges sum to zero - reoccurring events form peak
\item found multiple peaks: D mason from $K\pi$ decay and two relatively slightly shifted peaks in both $K\pi$ and $\pi \pi$
\item applied D mason cut: reject all events with D mason mass $\pm 30 ~MeV/c^2$
\item no need to exclude muon tracks misidentified as pions - excluded muons from the start
\item \textit{why do muon get easily misidentified as pions?} - similar mass
\item \textit{check that none of the resonance peaks have c quarks}
\end{itemize}
\end{itemize}

\section{Week 2}
\begin{itemize}
\item Fitting:
\begin{itemize}
\item Normalized Gaussian to signal and combinatorial background
\item Exponential to background
\item subtracted background and combinatorial background from histogram
\item separated $B^+$ and $B^-$ decays by considering $K$ charge and plotted them individually
\item Gaussian integral divided by bin width gave counts in each peak
\item from this we could find value CP asymmetry - consistent with 0
\end{itemize}
\item{Dalitz plots:}
\begin{itemize}
\item created dalitz plots for simulation data
\item created dalitz plots for all data - including D mason. Clearly saw this on the plot
\item cut regions of $5284.37 \pm 40~MeV/c^2$ of B mason mass
\item did this individually for $B^\pm$
\item rough background subtraction by taking the background from the B mason mass data, just above the peak and subtracting that
\item background data is from higher mass region and therefore slightly larger on the Dalitz plot - hence we get negative entries on the upper edge
\item floor histograms to zero

\end{itemize}
\end{itemize}

\section{Week 3}
\begin{itemize}
\item{Dalitz plots:}
\begin{itemize}
\item integrate background function under peak and the background used for subtraction and scale background used for subtraction
\item variable bin size so that fluctuations of number of bins across dalitz plots for $B^\pm$ are minimal - could have done by optimization, but only did it by eye to save time
\item calculate error in each bin first using the \verb| hdalitzass->Sumw2();| function and then the\\ \verb| hdalitzass->GetBinError(a,b) | function
\item cut all errors $> 0.4$
\item set all bins to zero which have 0 entries in either of $B^\pm$ plots
\item produce significance plot: asymmetry / error - this shows us the most trustworthy regions
\item produced asymmerty plot with error cuts
\end{itemize}

\item{CP violation}
\begin{itemize}
\item best region found was $m_{K\pi}^2 < 15~GeV^2/c^4$ and $0~GeV^2/c^4 < m_{\pi\pi}^2 < 0.6~GeV^2/c^4$
\item only for that region produced three body decay histograms with errorbars - both for $B^\pm$
\item see a clear difference in the peak - $B^-$ decay leads to a much larger signal than $B^+ decay$
\item Now do a plot for $B^+$ and $B^-$ combined and do a fit
\item now for individual plots, take a 2 sigma range around the mean and the width of the combined fit and set this as a range for the individual fits - there is no reason why width and mean should be different
\item found a good fit, which represents data well and a asymmetry value consistent with the value found in the paper
\item analysed region 2 : $21~GeV^2/c^4 < m_{K\pi}^2 < 25~GeV^2/c^4$ and $1.2~GeV^2/c^4 < m_{\pi\pi}^2 < 3~GeV^2/c^4$
\item found negative asymmetry, as expected
\item here we can not see and 4body background.
\item had to add additional restrictions to make the fitting work. these included setting amplitude and sigma of the 4body gaussian to above 0
\end{itemize}
\end{itemize}

\section{Week 4}
\begin{itemize}
\item changed standard error that assumes poisson, to error including errors from fit
\item{Systematic errors}
\begin{itemize}
\item The decay $B^{\pm} \rightarrow J/ \psi K^{\pm}, J/\psi \rightarrow \mu^+\mu^-$? serves as a control channel for $B^{\pm} \rightarrow h^{\pm}h^+h^-$ decay modes. Since it has negligible CP violation, the raw asymmetry observed in $B^{\pm} \rightarrow J/ \psi K^{\pm}$ decays is entirely due to production and detection asymmetries.
\begin{itemize}
\item remove all cuts except for probK > 0.9 and introduce cut $3050~MeV/c^2 < m_{\pi\pi} < 3150~MeV/c^2$, which is the $J/ \psi$ mass
\item we can write all asymmetries as $A_{CP} = A_{raw} - A_{\Delta}$, where $A_{CP}$ is the desired result, $A_{raw}$ is what we measure and $A_{\Delta}$ is a correction term
\item we can estimate this correction term for $J/ \psi$ by measuring the asymmetry and subtracting the known value from PDG, errors added in quardature
\item this correction can be subtracted from all measurements as it represents detection/production asymmetry and the error on $A_{\Delta}$ is assigned as a type of systematic error
\end{itemize}
\item Magnet UP/DOWN:
\begin{itemize}
\item first created asymmetry dalitz plots for magnet up and down individually with individual backgrounds
\item then we created a dalitz plot of significance : difference between the magnet polarity data divided by errors added in quadrature
\item took all the bins and plotted a 1D histogram of the significance in each bin
\item found that this follows a Bell curve, consistent with mean 0, suggesting that this systematic error is not significant for this analysis, the statistical error is much larger
\item Analysing the significance of the asymmetry difference, for the different magnet polarities, in every bin in the Dalitz plot, it was shown that this follows a normal distribution, as expected for data under no systematic influence.
\end{itemize}
\item different models:
\begin{itemize}
\item took global asymmetry data and fit 4 models to the data
\item functions include: normalised gaussian, linear, exponential, landau
\item got different asymmetry from each
\item compute standard deviation from these  to get systematic error
\item repeat for every A
\item A is still just from one model but get a systematic error from this
\end{itemize}
\end{itemize}
\item{Resonance state cut}
\begin{itemize}
\item took a region of $\rho(770) \pm 100~MeV/c^2$ and cut this from the data and the we cut everything but this region
\item our region 1 in the dalitz plot contains $\rho(770)$ and therefore we saw a corresponding effect here.
\item when removing the region the asymmetry decreased drastically - also a larger error due to the fitting getting worse
\item on the contrary cutting everything but this region lead to a very large asymmetry
\item this suggests that CP asymmetry is caused by this resonance state
\item barely enough data points were available
\end{itemize}
\end{itemize}

\section{Improvements}
\begin{itemize}
\item more sophisticated background and signal fitting - maybe with physical / statistical reasoning behind it; this would reduce systematic uncertainty
\item Dalitz plot optimisation - now we could have possible misidentification of certain regions
\item further investigation on momentum dependence of systematic uncertainties - would affect some local regions more than others if present
\end{itemize}

\section{Results}
~\\[5mm]

\LARGE Global asymmetry\\[2mm] $~~~A_{CP} = -0.004~\pm0.013\textrm{(stat)}~\pm0.007\textrm{(syst)}~\pm0.007\textrm{($J/\Psi$)}$\\[5mm]
\LARGE Local negative asymmetry\\[2mm] $~~~A_{CP} = -0.225~\pm0.009(stat) ~\pm0.018(syst)~\pm0.007\textrm{($J/\Psi$)}$\\[5mm]
%CHANGE!!!!!!!!
\LARGE Local positive asymmetry\\[5mm] $~~~A_{CP} = ~0.729~\pm0.051\textrm{(stat)}~\pm0.048\textrm{(syst)}~\pm0.007\textrm{($J/\Psi$)}$\\[5mm]

\LARGE $670< M_{\pi\pi} < 870~[\textrm{MeV/c$^2$}]:$\\[1mm]
$~~~A_{CP} = 0.746~\pm0.042\textrm{(stat)}~\pm0.048\textrm{(syst)}~\pm0.007\textrm{($J/\Psi$)}$\\[5mm]
\LARGE $ M_{\pi\pi} < 670,~M_{\pi\pi} > 870~[\textrm{MeV/c$^2$}]:$\\[1mm]
$~~~A_{CP} = 0.452~\pm0.153\textrm{(stat)}~\pm0.048\textrm{(syst)}~\pm0.007\textrm{($J/\Psi$)}$\\[5mm]








\end{document}
